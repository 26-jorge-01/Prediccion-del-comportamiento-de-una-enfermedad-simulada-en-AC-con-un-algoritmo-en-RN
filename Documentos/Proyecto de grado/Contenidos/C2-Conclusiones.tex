\chapter{Conclusiones}

% Al realizar este trabajo se comprendió la manera en que se define la distancia $C^0$,\,la distancia de \textit{Hausdorff},\, la distancia \textit{Gromov-Hausdorff}. Así mismo, cómo se pueden calcular distancias entre objetos en donde estas métricas estén definidas usando las diversas caracterizaciones de distancia, así como propiedades que éstas o los objetos (conjuntos, espacios, funciones) poseen. Posterior a eso, fue posible entender qué motiva y cómo funciona la denominada distancia $GH^0$. Ya con estos conceptos claros, se logró comprender de qué manera difieren o son similares ciertos espacios y/o funciones en términos de qué tan cercanos son con respecto a una forma de definir esa distancia. Luego de eso, fue posible a través del entendimiento del concepto de estabilidad topológica de Walters \cite{WalPOTP}, estudiar la propuesta de $GH$-estabilidad topológica dada por Arbieto \& Morales en \cite{AM}.\\

% Las conclusiones que se pueden enunciar de este trabajo son:
% \begin{itemize}
%     % \item Mediante la introducción de métricas o pseudo métricas es posible estudiar y entender el comportamiento de espacios y sistemas, en el sentido de  comparar con elementos cercanos. A su vez, estudiar fenómenos de convergencia y estabilidad.
%     \item Todo homeomorfismo topológicamente estable es isométricamente estable. Sin embargo el recíproco es falso en general.
%     \item Un homeomorfismo topológicamente $GH$-estable es isometricamente estable. Esto quiere decir que una función $GH$-estable ante pequeñas perturbaciones no solo produce una función con una estructura dinámica, sino que además la dinámica misma de la perturbación es muy cercana. 
%     \item Existe un homeomorfismo el cual es topológicamente estable pero no es topológicamente $GH$-estable. Por ende, la $GH$-estabilidad es una condición distinta de estabilidad.
%     \item Todo homeomorfismo topológicamente $GH$-estable en el circulo es topológicamente estable.
%     \item Todo homeomorfismo  expansivo con la P.O.T.P  de un espacio métrico compacto es topológicamente $GH$-estable (extendiendo el resultado dado por Walters).
%     \item Toda función continua positivamente expansiva con la $P.O.T.P_+$ de un espacio métrico compacto es topológicamente $GH-$estable (Caso no invertible).
%     \item Lo desarrollado en este trabajo sugiere que en espacios compactos la $GH$-estabilidad es una condición más fuerte que la estabilidad topológica. Sin embargo, esto aún no ha sido probado.
% \end{itemize}