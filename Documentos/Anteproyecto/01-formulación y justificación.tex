Las matemáticas han sido pieza fundamental en el estudio epidemiológico, el cual busca predecir el comportamiento de una enfermedad para mitigar su impacto, tomando medidas de control y prevención. 

Un punto de partida es el modelo descrito por Kermack y McKendrick en 1927, también conocido como modelo SIR, en el que se establecen relaciones entre tres estados de una enfermedad (Susceptible-Infectado-Recuperado) y se implementan los conceptos de tasa de contagio y de recuperación \cite{malariaSIR}. Desde entonces se han desarrollado múltiples variaciones sobre el modelo SIR, con el objetivo de analizar diferentes tipos de enfermedades de una manera más precisa, considerando por ejemplo diferentes estados, tasas de natalidad y mortalidad, entre otros \cite{diego2010}.

Por otra parte y debido a los avances tecnológicos de las últimas décadas, se han desarrollado modelos y simulaciones que permiten analizar características que no eran posibles con los modelos anteriormente mencionados. Por ejemplo, patrones de movilidad \cite{colaGNN, epidemiologicalNeuralNetwork}, disminución en la cantidad de contagios debido a un aislamiento preventivo \cite{stayHome}, contagios de individuo a individuo \cite{heterogeneousPopulation} e interacciones en masa \cite{combiningGraph, transfer2021}, la mayoría realizadas con una fuerte influencia de las redes neuronales y complejas, apoyadas fuertemente sobre la teoría de grafos. 

La inexistencia de un algoritmo predictivo que tenga en cuenta las interacciones más cercadas de cada individuo, es un problema que se ha evidenciado principalmente, a causa de la naturaleza de las bases de datos disponibles. A pesar del fuerte acercamiento de los modelos epidemiológicos a las dinámicas sociales con la llegada del coronavirus, se consideran interacciones a escalas que podrían ser mayores a las que se contagia la enfermedad, si bien existen modelos que analizan dinámicas de persona a persona, sus capacidades predictivas resultan limitadas por la naturaleza con la que se generan las bases de datos, como ocurre en \cite{combiningGraph,transfer2021}.

Teniendo en cuenta la aplicabilidad de los autómatas celulares y las cualidades predictivas de los modelos en redes neuronales, se pretende simular el comportamiento de una enfermedad y desarrollar un algoritmo que permita realizar pronósticos sobre su comportamiento, para responder a la pregunta: ¿Qué impacto tienen las relaciones sociales cercanas, en la propagación de una enfermedad?